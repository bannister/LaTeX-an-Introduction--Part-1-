\documentclass[11pt]{exam}

\usepackage{listings}
\usepackage{lipsum}
\usepackage{graphicx}
\usepackage{color}
\usepackage{multirow}


\lstdefinestyle{latexsty}{
	language={[LaTeX]TeX},
    basicstyle=\small\ttfamily,
    breaklines=true,
    breakindent=0pt, 
    numbers=left, stepnumber=1, numbersep=5pt,
    commentstyle=\color{red},
    showstringspaces=false,
    keywordstyle=\color{blue}\bfseries,
    morekeywords={align,begin},
    tabsize=2,
    frame=single,
}

\pagestyle{headandfoot}
\runningheader{\LaTeX: An Introduction (Part 1)}{Exercise 3}{10th February 2014}
\runningheadrule
\runningfooter{}{\thepage}{}
\runningfootrule

\title{Exercise 3 - Lists and Tables in \LaTeX}
\author{UGC `\LaTeX: An Introduction (Part 1)' Training Course}
\date{February 10th 2014}

\begin{document}
\maketitle

\fbox{\parbox{0.9\textwidth}{In this exercise, we will practice making lists and tables. You can do it in a separate document, or continue adding content to the document you used in the last exercise.}}

\vspace{0.2in}

\section{Lists}

\begin{questions}

\question
Add a simple list that looks like the one below to your document.

\begin{itemize}
	\item A list item
	\item Another list item
	\item Yet another list item
\end{itemize}


\question
Can you change the list you just made so that the items are numbered? How?

\question
Try creating a description list like the one below. Again, the text you use doesn't particularly matter.

\vspace{0.1in}
\begin{minipage}[ht!]{0.4\textwidth}
\begin{description}
	\item[First Paragraph] Some text about something, I'm not sure what
	\item[Second Paragraph] Some more text about something else that is different?
	\item[Third Paragraph] This isn't the same as the previous items
\end{description}
\end{minipage}

\question
Create a nested list. Can you nest lists of different type within each other?

\end{questions}

\section{Tables}

\begin{questions}

\question Make a simple table like the one below:

\begin{center}
\begin{tabular}{l | l | l}
    a & b & c \\
    d & e & f \\
    g & h & i \\
    j & k & l \\
\end{tabular}
\end{center}

\question Add a header row to your table so it looks like this:

\begin{center}
\begin{tabular}{l | l | l}
	1 & 2 & 3 \\
	\hline
    a & b & c \\
    d & e & f \\
    g & h & i \\
    j & k & l \\
\end{tabular}
\end{center}

\question
Create a table like the one below where the text in one of the columns is too large to fit on the page without wrapping:

\begin{center}
    \begin{tabular}{ | r | l | l | p{6cm} |}
    \hline
   	Number & Player & Position & Summary \\ \hline
	1 & Chris Weale & Goalkeeper & Experienced goalkeeper who moved to the Greenhous Meadow in July 2012 after three seasons at Championship club Leicester City. \\ \hline
	2 & Jermaine Grandison & Defender & The powerful defender, equally comfortable at right back or centre half, joined the club from Coventry City, initially on loan in January 2011, impressing enough to be offered a permanent deal.\\ \hline
	7 & Mark Wright & Midfielder & The Wolverhampton-born winger has emerged as a consistent goalscorer since joining Shrewsbury, registering double figures from out wide in each of his first two seasons at the Greenhous Meadow.\\ \hline
    \end{tabular}
\end{center}

\question
Create a table with a heading cell that spans multiple columns as below:

\begin{center}
\begin{tabular}{ |l|l| }
  \hline
  \multicolumn{2}{|c|}{Team sheet} \\
  \hline
  GK & Chris Weale \\
  DF & Jermaine Grandison \\
  DF & Joe Jacobson \\
  DF & Connor Goldson \\
  DF & Cameron Gayle \\
  MF & Mark Wright \\
  MF & Paul Parry \\
  MF & Aaron Wildig \\
  MF & Matt Richards \\
  ST & Marvin Morgan \\
  ST & Ryan Doble \\
  \hline
\end{tabular}
\end{center}
\clearpage

\question 
Create a complex table like the one below where some cells stretch over multiple columns and some stretch over multiple rows. (\emph{Hint - remember the \texttt{multirow} package})
\begin{center}
\begin{tabular}{ |l|l|l| }
\hline
\multicolumn{3}{ |c| }{Team sheet} \\
\hline
Goalkeeper & GK & Chris Weale \\ \hline
\multirow{4}{*}{Defenders} & LB & Jermaine Grandison \\
 & DC & Joe Jacobson \\
 & DC & Connor Goldson \\
 & RB & Cameron Gayle \\ \hline
\multirow{3}{*}{Midfielders} & MC & Mark Wright \\
 & MC & Paul Parry \\
 & MC & Aaron Wildig \\ \hline
Forward & FW & Matt Richards \\ \hline
\multirow{2}{*}{Strikers} & ST & Marvin Morgan \\
 & ST & Ryan Doble \\
\hline
\end{tabular}
\end{center}

%\question 
%Wrap one of your tables in a \texttt{table} environment. Add a label and a caption to it. Can you then refer to that table in the text with a cross reference?

%\question
%Experiment with the position specifier on the \texttt{table} environment to see how \LaTeX\ positions your floating table.

\end{questions}

\end{document}