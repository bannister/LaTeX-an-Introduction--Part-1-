\documentclass[11pt]{exam}

\usepackage{listings}
\usepackage{lipsum}
\usepackage{graphicx}
\usepackage{color}
\usepackage{multirow}
\usepackage{amsmath}
\usepackage{xfrac}
\usepackage[T1]{fontenc}


\lstdefinestyle{latexsty}{
	language={[LaTeX]TeX},
    basicstyle=\small\ttfamily,
    breaklines=true,
    breakindent=0pt, 
    numbers=left, stepnumber=1, numbersep=5pt,
    commentstyle=\color{red},
    showstringspaces=false,
    keywordstyle=\color{blue}\bfseries,
    morekeywords={align,begin},
    tabsize=2,
    frame=single,
}

\pagestyle{headandfoot}
\runningheader{\LaTeX: An Introduction (Part 1)}{Exercise 5}{10th February 201
4}
\runningheadrule
\runningfooter{}{\thepage}{}
\runningfootrule

\title{Exercise 5 - Maths in \LaTeX}
\author{UGC `\LaTeX: An Introduction (Part 1)' Training Course}
\date{February 10th 2014}

\begin{document}
\maketitle

\fbox{\parbox{0.9\textwidth}{In this exercise, we will practice inserting maths into our documents. Again, you can use a separate new document, or continue adding content to the document you used in the previous exercises. Remember to add the \texttt{amsmath} package to your document using the \texttt{{\textbackslash}usepackage} command}}

\vspace{0.1in}


\begin{questions}

\uplevel{}

\question
Add a few lines of text to your document. Within this text, use the \texttt{math} environment to typeset the following equation inline.

\begin{equation*}
	(a + b)(a + b) = a^2 + 2ab + b^2
\end{equation*}

\uplevel{Your result should look like this:

\vspace{0.1in}
\fbox{\parbox{\textwidth}{Some text that I am typing just so I can include this equation: \( (a + b)(a + b) = a^2 + 2ab + b^2 \) which I have now done.}}}

\question
Swap the \texttt{math} environment \texttt{begin} and \texttt{end} for \texttt{\textbackslash(} and \texttt{\textbackslash)}. Does it make a difference to the output? What if you use \$ \ldots \$ instead? 


\question
Use the \texttt{equation} environment to add the following equations to your document:


\[
a^3 + 6b - 12
\]

\[
\cos (2\theta) = \cos^2 \theta - \sin^2 \theta
\]

\[
\lim_{x \to \infty} \exp(-x) = 0
\]

\[
k_{n+1} = n^2 + k_n^2 - k_{n-1}
\]

\[
\frac{n!}{k!(n-k)!} 
\]

\[
\frac{\frac{1}{x}+\frac{1}{y}}{y-z}
\]

\[
3\times\sfrac{1}{2}=1\sfrac{1}{2}
\]

\[
\left|\sum_{i=1}^n a_ib_i\right|
\]

\[
\left(\sum_{i=1}^n b_i^2\right)^{1/2}
\]

%\question
%Add a label to one of the equations in your document, and reference it from within the text.

\question
Experiment with adding equations from your own research background. 

\end{questions}

\end{document}